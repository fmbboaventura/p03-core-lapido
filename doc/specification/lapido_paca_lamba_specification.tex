\documentclass{article}

\usepackage{ipprocess}
% Use longtable if you want big tables to split over multiple pages.
% \usepackage{longtable}
\usepackage[utf8]{inputenc} 
% Babel package is used to translate keywors to a specific language. 
% The option "brazil" defines portuguese-brazil as default language.
% Babel is also useful for hiphenation.
\usepackage[brazil]{babel}
\usepackage{mathtools}
% \usepackage[LGRgreek]{mathastext}

\sloppy

\title{MUSA Core Processor}

\graphicspath{{./pictures/}} 
\makeindex
\begin{document}
\DocumentTitle{Especificação do Projeto}
\Project{Processador LAPI DOpaCA LAMBA}
\Organization{Universidade Estadual de Feira de Santana}
\Version{Build 2.1}
 \capa

%%%%%%%%%%%%%%%%%%%%%%%%%%%%%%%%%%%%%%%%%%%%%%%%%%
%% Historico de Revisoes
%%%%%%%%%%%%%%%%%%%%%%%%%%%%%%%%%%%%%%%%%%%%%%%%%%
\section*{\center Histórico de Revisões}
  \vspace*{1cm}
  \begin{table}[ht]
    \centering
    \begin{tabular}[pos]{|m{2cm} | m{7.2cm} | m{3.8cm}|} 
      \hline
      \cellcolor[gray]{0.9}
      \textbf{Date} & \cellcolor[gray]{0.9}\textbf{Descrição} & \cellcolor[gray]{0.9}\textbf{Autor(s)}\\ \hline
      \hline
      \small xx/xx/xxxx & \small <Descrição> & \small <Autor(es)> \\ \hline
      \small 08/12/2015 & \small Teste & \small Filipe Boaventura \\ \hline
      \small xx/xx/xxxx &
      \begin{small}
        \begin{itemize}
          \item Exemplo de;
          \item Revisões em lista;
        \end{itemize}
      \end{small} & \small <Autor(es)> \\ \hline 
    \end{tabular}
  \end{table}

\newpage
  
\tableofcontents
\newpage

%%%%%%%%%%%%%%%%%%%%%%%%%%%%%%%%%%%%%%%%%%%%%%%%%%
%% Concent
%%%%%%%%%%%%%%%%%%%%%%%%%%%%%%%%%%%%%%%%%%%%%%%%%%
  \section{Introducão}
  \subsection{Propósito}
  O objetivo desse documento é definir as especificações para o processador \textbf{LAPI DOpaCA LAMBA}, incluindo o seu montador e simulador. Aqui são definidos os parametros de implementação do processador que incluem: modos de operação, arquitetura do conjunto de instruções (ISA), e a descrição dos registradoes e flags.
   Este documento está organizado da seguinte forma: Visão Geral na seção 2, Arquitetura do Conjunto de Instruções seção 3, Montador seção 4, Simulador seção 5.
  
  \subsection{Document Outline Description}
  Este documento está organizado da seguinte forma:
	
	\begin{itemize}
	  \item Section \color{black}{\ref{sec:design_overview}}: Apresenta uma visão geral do processador. 
	  \item Section \color{black}{\ref{sec:isa}}: Especifica a arquitetura do conjunto de instruções.
	  \item Section \color{black}{\ref{sec:register}}: Describes the general purpose registers.
	\end{itemize}
		
  \subsection{Acronyms and Abbreviations}
  
  \FloatBarrier
  \begin{table}[H]
    \begin{center}
      \begin{tabular}[pos]{|m{2cm} | m{9cm}|} 
	\hline
	\cellcolor[gray]{0.9}\textbf{Acronym} & \cellcolor[gray]{0.9}\textbf{Description} \\ \hline
	    RISC   & Reduced Instruction Set Computer \\ \hline
	    GPR    & General Purpose Registers \\ \hline
	    FPGA   & Field Gate Programmable Array \\ \hline
	    GPPU   & General Purpose Processing Unit \\ \hline
	    SPRAM  & Single-port RAM \\ \hline	    
	    HDL    & Hardware Description Language \\ \hline
	    CPU    & Central Processing Unit \\ \hline
	    ISA    & Instruction Set Architecture \\ \hline
	    ALU    & Arithmetic and Logic Unit \\ \hline
	    PC     & Program Counter \\ \hline
	    RF     & Flags Register \\ \hline
	    CST    & Constant Value \\ \hline
      \end{tabular}
    \end{center}
  \end{table}
  
  \newpage
  \section{Visão Geral}
  \label{sec:design_overview}
  O LAPI DOpaCA LAMBA é um processador RISC que apresenta um conjunto de 42 instruções organizadas em X grupos funcionais e possiu três modos de endereçamento. O processador possui 16 registradores de propósito geral em seu banco de registradores. O LAPI DOpaCA LAMBA suporta operações lógicas e aritiméticas básicas, excerto multiplicação e divisão. 
  
 \subsection{Perspectives}
    
  The MUSA 32-bit core release targets modern FPGA devices with embedded SPRAM memory, and is intended to be deployed as a GPPU soft IP-core. Its architecture is designed under MIPS original implementation, presenting a slightly reduced ISA.
  
  \subsection{Características Principais}
  
  \begin{itemize}
   \item Arquitetura de 32 bits;
   \item 16 registradores de propósito geral;
   \item Arquitetura de conjunto de instruções (ISA) RISC;
   \item ISA composta por 42 instruções;
   \item Organização de dados Big-Endian;
   \item Três modos de endereçamento: immediato, por registrador, e base-deslocamento;
   
  \end{itemize}

  \subsection{Non-functional Requirements}
  
  \begin{itemize}
   \item The FPGA prototype should run in a Terasic ALTERA Cyclone III (EP3C25F324) Development platform;
   \item The design must be described using Verilog-HDL;
   \item The tests must be writen in both Verilog-HDL or SystemVerilog;
   \item A set of test programs must be provided in order to validate the implementation;
  \end{itemize}

  
  \newpage
  \section{Conjunto de Instruções}
  \label{sec:isa}
  O Processador LAPI DOpaCA LAMBA possui 42 instruções de 32 bits de largura que estão organizadas nos seguintes grupos funcionais:
  \begin{itemize}
  \item Acesso à Memória (load e store);
  \item Constantes;  
  \item ALU;
  \item Transferências de controle (jump);
  \item NOPs;
  \item Condição de término.
  \end{itemize}
  
  No LAPI DOpaCA LAMBA, as operações são realizadas utilizando operandos localizados no banco de registradores. A memória principal é acessada apenas através das instruções load e store.
  
  \subsection{Instruções de Acesso à Memória}
  
    \begin{table}[H]
      \begin{center}
        \begin{tabular}[pos]{|c|c|c|} 
        \hline
			\cellcolor[gray]{0.9}\textbf{Op Código} & \cellcolor[gray] {0.9}\textbf{Mnemônico} & \cellcolor[gray] {0.9}\textbf{Operação} \\ \hline    		
           10100 &	load c, a & C = Mem[A]\\ \hline
		   10110 &   store a, b & 	Mem[A] = B \\ \hline
    	\end{tabular}
      \end{center}
    \end{table}
  
  \subsection{Instruções com Constantes}
  
  As instruções abaixo carregam constantes para um registrador de propósito geral.
  
    \begin{table}[H]
      \begin{center}
        \begin{tabular}{|c|c|c|} 
          \hline
          \cellcolor[gray]{0.9}\textbf{Formato} & \cellcolor[gray]{0.9}\textbf{Mnemônico} & \cellcolor[gray]{0.9}\textbf{Operação} \\ \hline
			I & C = Constante & loadlit c, Const \\ \hline
            II & C = Const8 | (C \& 0xff00) &	lcl c, Const8 \\ \hline
            II & C = (Const8 << 8) | (C \& 0x00ff) &	lch c, Const8 \\ \hline
 		\end{tabular}
      \end{center}
    \end{table}
    \begin{table}[H]
      \begin{center}
        \begin{tabular}[pos]{|c|c|} 
        \hline
			\cellcolor[gray]{0.9}\textbf{R} & \cellcolor[gray] {0.9}\textbf{Operação} \\ \hline    		
            0 & lcl \\ \hline
			1 & lch \\ \hline
    	\end{tabular}
      \end{center}
    \end{table}
    
  \subsection{Instruções ALU}
  
  O LAPI DOpaCA LAMBA realiza operações lógicas e aritméticas de 32-bits usando dois ou três operandos. As instruções ALU seguem o seguinte formato:
  
  \begin{table}[H]
      \begin{center}
        \begin{tabular}[pos]{|c|c|c|c|} 
          \hline
          \cellcolor[gray]{0.9} \textbf{Código} & \cellcolor[gray]{0.9} \textbf{Mnemônico} & \cellcolor[gray]{0.9}\textbf{Operação} & \cellcolor[gray]{0.9}\textbf{Flags Afetadas} \\ \hline
          00000 & zeros c   &   C = 0   &   Zero \\ \hline
          00001 & and c,a,b   &   C = A \& B   &   neg e zero \\ \hline
          00010 & andnota c,a,b & C = (\~A) \& B & neg e zero\\ \hline
          00011 & passa c,a & C = A  &  neg e zero\\ \hline
          00100 & xor c,a,b & C = A xor B & neg e zero\\ \hline
          00101 & or c,a,b & C = A | B & neg e zero \\ \hline
          00110 & nor c,a,b & C = !(A | B) & neg e zero \\ \hline
          00111 & xnor c,a,b & C = !(A xor B) & neg e zero \\ \hline
          01000 & passnota c,a & C = \~A & neg e zero \\ \hline
          01001 & ornotb c,a,b & C = A | \~B & neg e zero \\ \hline
          01010 & nand c,a,b & C = !(A \& B) & neg e zero \\ \hline
          01011 & ones c & C = 1 & nenhuma \\ \hline
          01100 & add c,a,b & C = A + B & rc = ra | rb \\ \hline
          01101 & addinc c,a,b & C = A + B + 1 & todas \\ \hline
          01110 & inca c,a & C = A + 1 & todas\\ \hline
          01111 & subdec c,a,b & C = A - B - 1 & todas \\ \hline
          10000 & sub c,a,b & C = A - B &  todas\\ \hline
          10001 & deca c,a & C = A - 1 & todas \\ \hline
          10010 & lsl c,a & C = shift lógico para a esquerda & neg, zero e carry \\ \hline
          10011 & lsr c,a & C = shift lógico para a direita & neg e zero \\ \hline
          10100 & asr c,a & C = shift aritmético para a direita & neg e zero \\ \hline
          10101 & asl c,a & C = shift aritmético para a esquerda & NÃO SEI \\ \hline
        \end{tabular}
      \end{center}
    \end{table}
    
  \subsection{Transferência de Controle}
  
  O LAPI DOpaCA LAMBA possui 5 instruções para transferencia de controle. Desvios condicionais com base no estado de uma das flags, desvio para um endereço contido num registrador, desvio para um endereço referente a um label e uma instrução para chamada de funções. 
  
   		 \begin{table}[H]
      \begin{center}
        \begin{tabular}[pos]{|c|c|c|c|} 
        \hline
			\cellcolor[gray]{0.9}\textbf{Formato} & \cellcolor[gray]{0.9}\textbf{Op Código} & \cellcolor[gray] {0.9}\textbf{Mnemônico} & \cellcolor[gray] {0.9}\textbf{Operação} \\ \hline    		
          I & 00	& Jump False 	& jf.cond Destino\\ \hline
		   I & 01 & Jump True &	jt.cond Destino \\ \hline
           II & 10 & Jump Incondicional &	j Destino\\ \hline
           III & 11 & Jump and Link &	jal b \\ \hline
           VI & 11 &	Jump Register &	jr b \\ \hline
    	\end{tabular}
      \end{center}
    \end{table}
   
   \begin{table}[H]
      \begin{center}
        \begin{tabular}[pos]{|c|c|} 
        \hline
			\cellcolor[gray]{0.9}\textbf{R} & \cellcolor[gray] {0.9}\textbf{Operação} \\ \hline    		
            0 & Jump and Link \\ \hline
			1 & Jump Register \\ \hline
    	\end{tabular}
      \end{center}
    \end{table}
    
    
    \begin{table}[H]
      \begin{center}
        \begin{tabular}[pos]{|c|c|c|} 
        \hline
			\cellcolor[gray]{0.9}\textbf{Op Código} & \cellcolor[gray] {0.9}\textbf{Mnemônico} & \cellcolor[gray] {0.9}\textbf{Condição} \\ \hline    		
           10100 &	.neg & Resultado ALU negativo\\ \hline
		   10110 &    	.zero & 	Resultado ALU zero \\ \hline
           10100 &	.carry & Carry da ALU\\ \hline
		   10110 &    	.negzero & 	Resultado ALU negativo ou zero \\ \hline
           10100 &	.true & Resultado ALU diferente de zero\\ \hline
		   10110 &    	.overflow & 	Resultado ALU overflow \\ \hline
    	\end{tabular}
      \end{center}
    \end{table}
  
  \subsection{CPU Load and Store Instructions}
  
 \FloatBarrier
  \begin{table}[H]
    \begin{center}
      \label{tab:load_store_instructions}
      \begin{tabular}[pos]{| l | l | l | l |} \hline 	
      \multicolumn{1}{|c|}{\cellcolor[gray]{0.9}\textbf{Mnemonic}} &
      \multicolumn{1}{c|}{\cellcolor[gray]{0.9}\textbf{Operands}} &  
      \multicolumn{1}{c|}{\cellcolor[gray]{0.9}\textbf{Realization}} &
      \multicolumn{1}{c|}{\cellcolor[gray]{0.9}\textbf{Description}} \\ \hline
	LW  & $RD,RS1,I_{16}$ & $RD$ \textleftarrow $MEM[RS1+I_{16}]$ & Load word from data memory.	\\ \hline
	SW  & $RS1,RS2,I_{16}$ & $MEM[RS1+I_{16}]$ \textleftarrow $RS2$ & Store word into data memory. 		\\ \hline
      \end{tabular}
    \end{center}
  \end{table}  
  
  
  \subsection{Computational Instructions}
  
  The MUSA provides 32-bit arithmetic operations. The computational instructions uses ALU two-operand instructions. These are signed versions of the following operations:

  \begin{itemize}
   \item Add
   \item Subtract
   \item Multiply
   \item Divide
  \end{itemize}
  
 \FloatBarrier
  \begin{table}[H]
    \begin{center}
      \begin{tabular}[pos]{| l | l | l | m{5cm} |} \hline 	
      \multicolumn{1}{|c|}{\cellcolor[gray]{0.9}\textbf{Mnemonic}} &
      \multicolumn{1}{c|}{\cellcolor[gray]{0.9}\textbf{Operands}} &
      \multicolumn{1}{c|}{\cellcolor[gray]{0.9}\textbf{Realization}} &
      \multicolumn{1}{m{5cm}|}{\cellcolor[gray]{0.9}\textbf{Description}} \\ \hline
	ADD  	& $RD$, $RS1, RS2$ 	& $RD$ \textleftarrow $RS1 + RS2$ 	& Add two words. 			\\ \hline
	SUB 	& $RD$, $RS1, RS2$ 	& $RD$ \textleftarrow $RS1 - RS2$ 	& Subtract two words. 		\\ \hline
	MUL 	& $RD$, $RS1, RS2$ 	& $RD$ \textleftarrow $RS1 * RS2$ 	& Multiply two words.		\\ \hline
	DIV   	& $RD$, $RS1, RS2$ 	& $RD$ \textleftarrow $RS1 / RS2$ 	& Divide two words.		\\ \hline
	AND  	& $RD$, $RS1, RS2$ 	& $RD$ \textleftarrow $RS1 \odot RS2$ 	& Logical AND.		\\ \hline
	OR 	    & $RD$, $RS1$ 	    & $RD$ \textleftarrow $RS1 \oplus RS2$ 	& Logical OR.	\\ \hline
	ADDI  	& $RD$, $RS1, I_{16}$ 	& $RD$ \textleftarrow $RS1 + I_{16}$ 	& Add a stored word and an \mbox{immediate} value. \\ \hline
	SUBI 	& $RD$, $RS1, I_{16}$ 	& $RD$ \textleftarrow $RS1 - I_{16}$ 	& Subtract a stored word and an immediate value.\\ \hline
	ANDI 	& $RD$, $RS1, I_{16}$ 	& $RD$ \textleftarrow $RS1 \odot I_{16}$ 	& Logical AND of a stored word and an immediate value		\\ \hline
	ORI   	& $RD$, $RS1, I_{16}$ 	& $RD$ \textleftarrow $RS1 \oplus I_{16}$ 	& Logical OR of a stored word and an immediate value.		\\ \hline	
	CMP 	& $RD$, $RS1$ 	    & -- 					            & Compares $RD$ and $RS1$ and set the flags.\\ \hline
	NOT 	& $RD$ 	            & $RD$ \textleftarrow $\sim RD$     & Logical NOT.	\\ \hline
      \end{tabular}
    \end{center}
  \end{table} 
  
  \subsection{Jump/Branch Instructions}

  \FloatBarrier
    \begin{table}[H]
      \begin{center}
		\begin{tabular}[pos]{| l | l | l | l |} \hline 	
	\multicolumn{1}{|c|}{\cellcolor[gray]{0.9}\textbf{Mnemonic}} &
	\multicolumn{1}{c|}{\cellcolor[gray]{0.9}\textbf{Operands}} &
	\multicolumn{1}{c|}{\cellcolor[gray]{0.9}\textbf{Characteristic}} &
	\multicolumn{1}{c|}{\cellcolor[gray]{0.9}\textbf{Description}} 			\\ \hline
	  JR  	& $R$ 		& Unconditional & Jump to destination. 			\\ \hline
	  JPC 	& $I_{28}$ 	& Unconditional & Jump to destination PC-relative. 	\\ \hline
	  BRFL 	& $RF$, $CST$ 	& Conditional 	& Jump to destination if $RF == CST$.\\ \hline
	  CALL   	& $R$		& Unconditional & Subroutine call.	\\ \hline
	  RET  	& -- 		& Unconditional & Subroutine return 			\\ \hline
	\end{tabular}
      \end{center}
    \end{table} 
  
  \subsection{No Operation Instruction}
  The \textit{No Operation} instruction (\texttt{NOP}) is used to control the instruction flow or to insert delays (stalls) into the datapath, such as when computing the result of a jump/branch instruction. When using a NOP instruction after a branch/jump instruction it is so named a \textbf{branch delay slot}.
  
  \subsection{End of Opperations}
  The HALT instruction (system stop) must be implemented as a \texttt{L: j L} (a unconditional branch to the current address)
  
  \section{Internal General Purpose Registers}
  \label{sec:register}
  The current MUSA architecture provides 32 fixed-point general purpose registers of 32 bits: $R_0$ to $R_{31}$. The $R_0$ have special use for the hardware always returning zero, no matter what software attempts to store to it.

\end{document}
